\documentclass{article}
\usepackage[utf8]{inputenc}
\usepackage[russian]{babel}
\usepackage{amsmath}
\usepackage{listings}
\usepackage{xcolor}

\title{Block-by-block description of MATLAB code}
\author{}
\date{}

% Code style
\lstset{
	language=Matlab,
	backgroundcolor=\color{gray!10},
	basicstyle=\ttfamily\small,
	frame=single,
	breaklines=true,
	keywordstyle=\color{blue},
	commentstyle=\color{green!60!black},
	stringstyle=\color{red},
	showstringspaces=false,
	numbers=left,
	numberstyle=\tiny\color{gray}
}

\begin{document}
	
	\maketitle
	
	\section*{Block-by-block description}
	
	\subsection*{1. Setting convergence and stagnation thresholds}
	
	\textbf{Convergence threshold (target residual norm):}  
	This value defines when the optimization is considered converged.
	
	\begin{lstlisting}
		target_residual_norm = 1e-3;
	\end{lstlisting}
	
	\vspace{1em}
	
	\textbf{Stagnation tolerance:}  
	If residual norm does not change significantly (difference less than this value), the algorithm is considered stuck.
	
	\begin{lstlisting}
		stagnation_tol = 1e17;
	\end{lstlisting}
	
	\subsection*{2. Initialization of initial guess}
	
	\textbf{Initial parameter vector:}  
	Pack parameters and fields into a vector.
	
	\begin{lstlisting}
		x0 = packFieldsAndInterface(fields, params.M, params.N, length(params.c_init), params.c_init);
	\end{lstlisting}
	
	\subsection*{3. Definition of anonymous residual function}
	
	\textbf{Anonymous function computing residual vector:}
	
	\begin{lstlisting}
		fun = @(x) computeResidualVector(x, params, dx, dy, X, Y, x_interface);
	\end{lstlisting}
	
	\subsection*{4. Optimization options}
	
	\textbf{Options for \texttt{lsqnonlin}:}  
	- detailed display,  
	- max iterations 100,  
	- function tolerance 1e-6.
	
	\begin{lstlisting}
		options = optimoptions('lsqnonlin', ...
		'Display', 'iter-detailed', ...
		'MaxIterations', 100, ...
		'FunctionTolerance', 1e-6);
	\end{lstlisting}
	
	\subsection*{5. Initialization for restart loop}
	
	\textbf{Variables to control restarts and residual norms:}
	
	\begin{lstlisting}
		max_restarts = 10;
		restart_count = 0;
		resnorm = Inf;
		previous_resnorm = Inf;
	\end{lstlisting}
	
	\subsection*{6. Restart optimization loop}
	
	\textbf{Run optimization while residual norm is above target and restarts are below maximum:}
	
	\begin{lstlisting}
		while resnorm > target_residual_norm && restart_count < max_restarts
		fprintf('Run #%d\n', restart_count + 1);
		
		[x_opt, resnorm, residual, exitflag, output] = lsqnonlin(fun, x0, [], [], options);
		
		fprintf('resnorm after iteration: %.6e\n', resnorm);
		
		delta_resnorm = abs(previous_resnorm - resnorm);
		if delta_resnorm < stagnation_tol
		fprintf('resnorm change too small: Δ = %.2e < %.2e → restart\n', delta_resnorm, stagnation_tol);
		end
		
		previous_resnorm = resnorm;
		x0 = x_opt;
		restart_count = restart_count + 1;
		
		save('optimization_intermediate.mat', 'x0', 'resnorm', 'restart_count');
		end
	\end{lstlisting}
	
	\subsection*{7. Final output}
	
	\textbf{Check if convergence was successful:}
	
	\begin{lstlisting}
		if resnorm <= target_residual_norm
		fprintf('Convergence achieved: resnorm = %.6e\n', resnorm);
		else
		fprintf('Accuracy not reached after %d restarts. Final resnorm = %.6e\n', restart_count, resnorm);
		end
	\end{lstlisting}
	
	\subsection*{8. Unpacking solution}
	
	\textbf{Convert optimized vector back to fields and parameters:}
	
	\begin{lstlisting}
		[fields_opt, c_opt] = unpackFieldsAndInterface(x_opt, params.M, params.N, length(params.c_init));
	\end{lstlisting}
	
\end{document}
